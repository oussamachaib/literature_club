\documentclass[a4paper,11pt]{article}
\usepackage[utf8]{inputenc}
\usepackage{fullpage}
\usepackage{graphicx}
\usepackage{wrapfig}
\usepackage{caption}
\usepackage{subcaption}
\usepackage{appendix}
\usepackage{tcolorbox}
\tcbuselibrary{theorems}

\newtcbtheorem[number within=section]{theo}{}%
{colback=green!5,colframe=green!35!black,fonttitle=\bfseries}{th}

\title{
	Literature notes
}
\author{\small Oussama Chaib}
\date{\small October 2021}

\begin{document}
	\maketitle
%	\tableofcontents
%	\pagebreak
\section{Turbulent burning rates of methane and methane–hydrogen mixtures }
\textit{Link:} Simone's Bibtex/Burn rate, turbulent velocity, consumption speed/1-s2.0-S0010218009000431-main.pdf\\
\textit{DOI:} http://dx.doi.org/10.1016/j.combustflame.2009.02.001\\ \\

\subsection{Introduction}
\begin{itemize}
	\item Potential for using existing natural gas infrastucture to transport a mixture of methane and hydrogen from production site to end user. Hydrogen can then be extracted from the mixture and used to power fuel cells or burned directly.
	\item Challenges: Safety implications on client and infrastructure (clients would receive a mixture of methane and hydrogen, current infrastructure designed for methane transport).
	\item Some properties of hydrogen flames:
	\begin{itemize}
		\item More reactive than natural gas.
		\item Laminar burning velocity: $S_l (H_2)  \approx 5.8 S_l (CH_4)$ (at stoichiometry)
		\item Wider flammability limit.
	\end{itemize}
	\item Scope: Identify and quantify the level of risk associated with accidental released of natural gas-hydrogen mixtures in \textbf{turbulent} events (most likely). To formulate and validate explosion and ignition models, turbulent and laminar burning velocities are necessary.
	\item \textbf{Turbulent flame velocity ($u_t$):} depends on:
	\begin{enumerate}
		\item Flow field: RMS turbulent velocity + eddy length scale.
		\item Flame chemistry: Laminar burning velocity ($u_l$), laminar flame thickness ($\delta_l = v/u_l$), sometimes a Lewis/Markstein number.
		\item Both the elements aforementioned are also influenced by temperature and pressure (as one would imagine).
	\end{enumerate}
	\item These effects on $u_t$ are usually compressed into a compact form involving dimensionless numbers (usually $u_t/u' = f(K) = g(Da)$ where K is the Karlovitz stretch factor and Da is Damköhler's number). Both K and Da relate the chemical + turbulent eddy lifetimes.
	
\end{itemize}


\end{document}
